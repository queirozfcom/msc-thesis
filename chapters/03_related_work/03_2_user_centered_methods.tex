\section{User-centered Methods}\label{section:user_centered_methods}

\subsection{Association Rule Mining}

\cite{frias-martinez_etal_2008}

\subsection{Neighbour-based Methods}

\cite{jaeschke_etal_2007} Resource-based or tag-based CF

\cite{marinho_schmidt-thieme_2008}

\cite{chidlovskii_2012} Propagates tags from content neighbours (and neighbours of neighbours) using LNP (linear neighbourhood propagation)

\cite{mistry_sen_2012} User-based CF

\cite{deng_etal_2013} Simple user-based and resource-based tag propagation from neighbours.

\cite{haifeng_etal_2017} 
User-based CF. Users are modelled with vectors over the space of tags used. Users are also clustered (K-Means)

\subsection{Graph-based Methods}

\cite{jaeschke_etal_2007} FolkRank 

The FolkRank vectors is taken as the difference between two computations of PageRank: one without a preference vector and one with a preference vector
The preference vector is designed in a way to give more weight to specific elements (if the goal is to recommend tags given a resource, the preference vector should give more weight to that resource)
(these explanations were in Ramezani 2011)

\cite{si_etal_2009} Diffusion rank, it's a generalization of PageRank

\cite{liu_fang_2010} Bipartite graph, random walk.

\cite{ramezani_2011} Adapted PageRank, with directed edges. Claims performs better than FolkRank.

\cite{gueye_etal_2014} Propagates tags from neighbour (not just direct neighbours) users. User similarity is measured by how similar users' tag frequencies are.
(new connections between indirect neighbours are also added by the algorithm)


\subsection{Matrix Factorization Methods}

\cite{symeonidis_etal_2008} HOSVD

\cite{rendle_schmidt-thieme_2010} Pairwise Interaction Tensor Factorization (PITF). It's a more efficient way to factorize 3D tensors.

\cite{harvey_etal_2010} LDA on the resource-tag matrix, weighted by user preferences.

\cite{symeonidis_etal_2010} HOSVD and Kernel-SVD on full 3D tensor

\cite{yang_etal_2012} Variant of HOSVD called HOSVD+, that uses a probabilistic interpretation and gives meaningful predictions for missing data imputation.

\cite{yao_etal_2017}
They model the folksonomy data as a Joint LDA model on both the resource-tag and on the user-tag matrices, jointly.

\subsection{Other}

\cite{hu_etal_2010} Bayesian approach.

\cite{krestel_fankhauser_2010} they train simple recommenders then joins them together via model averaging/voting.

\cite{yin_etal_2010} Bayesian approach.