\chapter{Introduction}\label{chap:intro}

In this chapter we will introduce the subject matter of this dissertation, namely Tag Prediction under Social Tagging Systems (STSs) and provide some insight into the problem scope and our line of research.

\section{Motivation}\label{section:intro_motivation}

The motivation for this work is twofold. 

Firstly, it is easy to see that naïve methods of organization may be hard to use in complex, real-world systems. Tags are one way to help users and administrators better organize and reason about concepts and/or resources in many such systems.

Take the following example: You organize scientific articles into folders and you have a new article called \textit{The History of Football in Europe}. Should you place it under \textit{"history"}, \textit{"sports"} or \textit{"Europe"}? Maybe place a copy under each folder? Create a new folder called \textit{"Europe\_Sports"} instead?\footnote{Note that similar classification schemes have already been in use for some time in places such as libraries. Some noteworthy examples include the Dewey Decimal System and the Library of Congress Classification methodologies, both in use for more than a century.} Clearly, assigning multiple labels or \textit{tags} to each resource is an elegant way of solving this problem.

Secondly, it is well established that suggesting tags to users promotes faster convergence to a common vocabulary \citep{marlow_etal_2006,hassan_etal_2009, dattolo_etal_2010} and increases the likelihood that resources are tagged \citep{dattolo_etal_2010,floeck_etal_2010}. Additionally, it has been claimed that identifying good tags from a set of recommended tags is orders of magnitude less demanding than coming up with good tags without intervention \citep{marinho_etal_2012}.

Since there are massive amounts of data available from Social Tagging Systems (STSs), it's natural to think of a data-driven, machine-learning based method to pursue that goal. In addition to being a worthy research problem from a theoretical standpoint, good tag prediction techniques could also effectively help people in the real world navigate these online communities.

\section{Problem scope}\label{section:intro_problem}

When considering Social Tagging Systems, one can envision many different problems and areas where scientific knowledge and research could be put to use. For this reason, we chose to address the problem of how to correctly predict which tags will be used to describe a given textual object in such a system. 

Since this problem touches upon many areas of scientific knowledge, such as machine learning, natural language processing (for textual resources), computer vision (similarly, for images and visual objects), recommender systems and so on, it is necessary to further limit our scope in a more precise manner.

Firstly, one may create a distinction between \textbf{broad and narrow folksonomies}.\footnote{Following commonly-cited sources such as \cite{wal_2005_broad_and_narrow}} \textbf{Broad folksonomies} are those where not just a resource's owner but the whole community of users may assign tags to any one resource available on the system. \textbf{Narrow folksonomies}, on the other hand, only allow items to be tagged once, by the person who has first added that particular item to the system (i.e. that item's owner). A summary can be seen on the following table:

\renewcommand{\arraystretch}{1.5}

\begin{table}[!h]
\centering
\caption{Broad and narrow folksonomies}
\begin{tabular}{|m{4cm}|m{8cm}|}
\hline
Broad folksonomies & Anyone can assign tags to any resource, and all tags are viewable by everyone using the system. \\
\hline
Narrow folksonomies & Only the owner of a given resource may assign tags to it. Other users can view them but cannot add their own.\\
\hline
\end{tabular}
\label{tab:broad_vs_narrow}
\end{table}

Broad folksonomies exhibit more diversity and richness of information, not to mention sheer scale, which makes them more amenable to analysis by data-driven methods, such as machine learning. More concretely, it has been suggested that a shared, global vocabulary of tags cannot be observed in narrow folksonomies \citep{schifanella_etal_2010}. 

Secondly, many authors \citep{illig_etal_2011,song_etal_2011} have made a distinction between \textbf{resource-centered user-centered approaches}. This refers to the fact that some approaches take user information into account when performing tag prediction (user-centered) while others take a global view, giving the same predictions for every user (resource-centered). Note that other authors, (\textit{e.g.} \cite{zhang_etal_2014} and \cite{hu_etal_2010} refer to these two types as \textit{personalized} and \textit{collective} tag recommendation). The following table summarizes these differences:

\begin{table}[!h]
\centering
\caption{Tag prediction approaches, classified according to the information they use}
\begin{tabular}{|p{3.5cm}|p{10cm}|}
\hline
Resource-centered approaches & Only information regarding the resources themselves is used to build the tag prediction mechanism. For a given resource, the same predictions are displayed for every user. \\
\hline
User-centered approaches & In addition to information about the resources, user-specific data (\textit{e.g.} users' tagging history and profile) is leveraged for suggesting tags to be used at tag assignment time.\\
\hline
\end{tabular}
\label{tab:resource_vs_user_centered}
\end{table}

In this work, have chosen to limit our scope to \textbf{resource-centered approaches to predicting tags in broad folksonomies}. This is due to the previous reasons and to the fact \citep{song_etal_2011} that user-centered approaches do not perform so well vis-a-vis resource-centered methods; the distribution of users and tags in broad folksonomies follow a power law and reusability of tags by each individual user is low. 

Resource-centered approaches more robust as we generally have much more information about resources than about users. This is particularly true with textual resources and broad folksonomies. Also, resource-centered methods have the added benefit of being able to work in the absence of user information, the so called \textit{cold start} problem.

More specifically, with regards to the chosen scope, we would like to inquire into the performance of different tag prediction techniques, as applied to different problem sets. We would like to be able able to answer questions such as:

\begin{itemize}
    \item How do different methods of multi-label ranking perform when applied to the same data?
    
    \item How do dataset characteristics such as the total number of resources, average number of tags per resource, etc, affect the outcomes?
    
    \item Does the type of feature representation used affect the outcomes for different methods? If so, how?

\end{itemize}

In parallel to this, we would also like to investigate the effect of the sparsity of features on a specific label ranking method, namely \textit{multi-instance multi-label SVM} (MIMLSVM), when applied to social tag prediction\footnote{For detailed  information on this, see section \ref{section:experiment_part_2}.}. 

This is an interesting method that was originally used for scene classification \citep{zhang_zhou_2007}. However, it was recently \citep{shen_etal_2009} adapted for text classification, with satisfactory results.

Given that there has been some work done on \textit{representation learning} for text \citep{bengio_etal_2003,efficientestimation,le_mikolov_2014} recently, we would like to investigate to what extent this particular method works when exposed to other kinds of text representations, namely \textit{dense} representations.

For this, we would like to answer questions such as:

\begin{itemize}
    \item Does MIMLSVM only work for the commonly-used bag-of-words representation?
    
    \item Do different types of dense representation affect the algorithm in different ways?
    
    \item Does the domain of the folksonomy under research affect the outcomes? If so, how?

\end{itemize}

\subsection{BTAS}

We would like to point out that most works in the literature do not take into account the number of times each tag has been assigned to a given resource. In other words, a \textit{binary} tag-assignment model or \textit{BTAS} \citep{illig_etal_2011} is used, whereby a tag assignment is equated with the fact that \textit{there exists at least one} user who assigned that tag to that resource. We follow that convention in this article.

\begin{definition}{BTAS}\footnote{Adapted from \cite{illig_etal_2011}.}

Let $\text{TAS}$ be all tag assignments made by all users $u \in U$, using tags $t \in T$ for resources $r \in R$:
\[ \text{TAS} = \{ (u,t,r) \in  U \times T \times R \ | \ \text{user} \ u \ \text{has assigned tag} \ t \ \text{to resource} \ r \} \]  

$\mathbf{BTAS}$ abstracts the user dimension away, considering instead a \textit{binary} tag assignment $(t,r)$ the existence of \textit{any} user $u$ having assigned tag $t$ to resource $r$:

\[ \text{BTAS} = \{(t,r) \in T \times R \ | \ \exists u \in U : (u,t,r) \in \text{TAS} \}  \]

\end{definition}

More information on these issues is given in \autoref{chap:social_tagging}.

\section{Methodology}\label{section:intro_methodology}

Here we will give a brief overview of the research methodology used in this work.

\subsection{Systematic Literature Review}\label{section:literature_review}

In order to add replicability and transparency to our literature review, we adopted principles from Systematic Literature Review, as defined in works such as \cite{baumeister_leary_1997} and \cite{bem_1995}.\footnote{Evidence such as screenshots of search results and the actual set of articles retrieved from each query can be provided upon request.}

We selected three reputable repositories of scientific articles and research pieces, namely IEEE-XPlore Digital Library\footnote{http://ieeexplore.ieee.org/Xplore/home.jsp}, ScienceDirect\footnote{https://www.sciencedirect.com/} and Scopus\footnote{https://www.scopus.com/}.

After initial contact with the subject matter of our work, we selected four sets of search terms, namely \textit{"collaborative tagging"}, \textit{"social tag prediction"}, \textit{"social tagging"} and \textit{"tag prediction"} and used those to search the titles, abstracts and contents of research pieces in the websites' databases.\footnote{These search terms were chosen because we believe they encompass a large part  of the available literature related to the topic of our work which we defined (for the purpose of this literature review) as \textbf{"Predicting or recommending tags in a social tagging environment"}.} 

We gathered and organized the results of the aforementioned queries; after removing duplicated entries, we had a collection of 2466 articles, book chapters or conference proceedings.

We read the abstracts of all 2466 pieces and, based on that, we selected 399 as being somehow related to the subject of our work, as explained above.

Out of these 399 relevant works, we further refined our set to 285 articles, by extending our analysis to the introduction and conclusion sections. This final list of 285 articles all contained information directly related to the task of predicting and/or recommending tags in a social tagging environment. They were all read in order to inform our research.\footnote{Other articles, which didn't feature in the search results, but were obviously relevant (based upon citation count for example), were also added and read.}


\section{Document structure}\label{section:intro_structure}

In this chapter we introduced the subject matter for this dissertation and the work methodology we will use.

In Chapter 2 we will give a brief overview of Social Tagging Systems and Folksonomies. In Chapter 3, related work is analyzed and compared. In Chapter 4 we will propose a solution to the problem we highlighted previously. In Chapter 5 we will explain the experiments conducted and analyze the results. Finally, in Chapter 6 we conclude this dissertation and provide pointers for future work and ways in which it can be extended and/or continued.



