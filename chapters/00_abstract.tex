\begin{abstract}
Este trabalho aborda o problema de predição de tags (rótulos) em sistemas de tagueamento colaborativo (Social Tagging Systems). É sabido que mecanismos de predição de tags em tais sistemas melhora a usabilidade dos mesmos aumenta a qualidade do vocabulário de tags. Com isso em mente,  verificamos a diferença no desempenho de métodos de predição de tags quando aplicados a dois datasets que se diferenciam quanto a número de tags por recurso, quantidade total de tags, quantidade total de recursos, etc. Também analisamos um método específico para predição de tags baseado na quebra de documentos em segmentos. Verificamos se o mesmo generaliza para representações densas de textos. Experimentos são realizados nestes dois conjuntos de dados e os resultados obtidos são relatados.
\end{abstract}