\begin{abstract}
Este trabalho apresenta duas propostas relacionadas a sistemas de tagueamento colaborativo baseados em recursos textuais. Primeiro, verificamos a diferença na performance de métodos de predição de tags quando aplicados a dois datasets que se diferenciam quanto a número de tags por recurso, quantidade total de tags, quantidade total de recursos, etc. Segundo, verificamos se um método para predição de tags baseado na quebra de documentos em segmentos generaliza para representações densas (não-esparsas de textos). Experimentos são realizados nestes dois conjuntos de dados e os resultados obtidos são relatados.
\end{abstract}